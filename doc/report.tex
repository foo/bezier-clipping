\documentclass[11pt,a4paper,oneside]{report}

\usepackage[utf8]{inputenc}
\usepackage[polish]{babel}
\usepackage{polski}
\usepackage{graphics}
\usepackage{amsmath}
\usepackage{mathrsfs}

\makeatother
\title{Znajdowanie zer wielomianu w postaci Beziera za pomocą przycinania sześciennego}
\author{Maciej Pacut}
\date{Wrocław 2010}


\begin{document}
\maketitle
\newpage

\section{Wstęp}

\section{Numeryczne znajdowanie zer}

\subsection{Znajdowanie jednego zera}

W niniejszej pracy będziemy rozpatrywać zbiór metod znajdowania zer o pewnych cechach wspólnych. Działanie tych metod polega na zawężaniu danego przedziału do momentu osiągnięcia zadanej dokładności lub wykluczenia istnienia zera w danym przedziale.

Metoda znajdowania jednego zera funkcji dostaje argumenty:
\begin{enumerate}
\item Pewien opis funkcji f, który umożliwia obliczenie wartości tej funkcji dla dowolnego argumentu.
\item Przedział początkowy $P_0$, czyli przedział w którym szukamy zera funkcji f; metoda nie znajdzie żadnego zera spoza tego przedziału.
\item Żądaną dokładność $\varepsilon$ znalezienia zera funkcji f.
\end{enumerate}

Metoda numerycznego znajdowania jednego zera funkcji f generuje ciąg przedziałów $\langle P_0, P_1, ..., P_n\rangle$ o własnościach:
\begin{enumerate}
\item Każdy przedział tego ciągu jest podprzedziałem swojego poprzednika:
$$P_k \subset P_{k-1}\hspace{1cm}(1 \leq k \leq n)$$
\item Jeśli poprzednik pewnego ciągu zawiera zero to ten przedział także zawiera to zero:
$$ x_0 \in P_{k-1}\Rightarrow x_0 \in P_k\hspace{1cm}(1 \leq k \leq n)$$
\item $P_n$ jest jedynym przedziałem krótszym niż $2 \varepsilon$.
\end{enumerate}

Jeśli w przedziale $P_0$ znaleziono zero, to wynikiem działania metody jest środek przedziału $P_n$. Skoro zero znajduje się w $P_n$ i jego długość jest mniejsza niż $2 \varepsilon$ to udało się zlokalizować zero funkcji f z błędem mniejszym niż $\varepsilon$. Jeśli nie znaleziono zera, $P_n$ jest przedziałem pustym.

Metody znajdowania zer zdefiniowane powyższym schematem różnią się sposobem konstrukcji ciągu $\langle P_0, P_1, ..., P_n\rangle$. Do kompletnego opisu dowolnej metody z tego zbioru wystarczy zdefiniować, w jaki sposób mając przedział $P_k$ konstruujemy przedział $P_{k+1}$.
z
Przykładem metody znajdowania jednego zera funkcji jest bisekcja. W bisekcji przedział $P_{k+1}$ otrzymuje się dzieląc $P_k$ na pół i jako $P_{k+1}$ wybierając tę połowę, w której jest możliwość wystąpienia zera. W przypadku, gdy w żadnej połowie nie ma możliwości wystąpienia zera, $P_{k+1}$ jest przedziałem pustym i metoda kończy działanie. W bisekcji należy określić realizację operacji określającej możliwość wystąpienia zera w danym przedziale (w danej połowie przedziału $P_k$). Przykładową taką operacją jest sprawdzenie, czy funkcja f ma w końcach sprawdzanego przedziału różne znaki.

! rysunek bisekcji

\subsection{Znajdowanie wielu zer}

Metod znajdowania jednego zera nie stosuje się do funkcji mających więcej niż jedno zero w przedziale $P_0$. Przykładowo, metoda bisekcji ma działanie niezdefiniowane w przypadku, gdy w obu połowach przedziału jest możliwość wystąpienia zera.

Zmodyfikujmy powyższy schemat znajdowania jednego zera. Zamiast generowania ciągu przedziałów, metoda szukania wielu zer generuje drzewo przedziałów. Drzewo przedziałów ma własności:
\begin{enumerate}
\item Korzeniem jest przedział początkowy $P_0$.
\item Każdy potomek jest podprzedziałem swojego rodzica.
\item Przedziały mające wspólnego rodzica nie mają części wspólnych.
\item Każde zero znajdujące się w przedziale znajduje się w także w pewnym przedziale potomnym.
\item Każdy liść $L$ jest jedynym przedziałem na ścieżce od korzenia do $L$, który ma długość mniejszą niż $2 \varepsilon$
\end{enumerate}

Wynikiem działania metody jest zbiór środków niepustych przedziałów znajdujących się w liściach drzewa przedziałów.

!! Rysunek drzewa przedziałów w bisekcji przy dwóch podprzedziałach, rysunek drzewa przedziałów zdegenerowanego do listy
!! Podpis do rysunku: W wypadku 1 zera drzewo moze sie degenerowac do listy, ktora mozna traktowac jako ciąg z poprzedniego schematu.

Przedstawiony został w powyższym schemacie pewien zbiór metod znajdowania jednego zera funkcji f. Metody w tym zbiorze różnią się realizacją funkcji generującej drzewo przedziałów.

Przykładem metody szukania wielu zer funkcji jest modyfikacja opisanej w poprzednim podrozdziale metody bisekcji. W przypadku gdy w obu połowach dzielonego przedziału jest możliwość wystąpienia zera, obie połowy są określane jako potomkowie przedziału.

\subsection{Szybkość metody znajdowania zer}

Szybkość znajdowania zer jest jedną z miar jakości metody. Możemy do pomiaru szybkości użyć komputera. Można zaprogramować metodę szukania zer w pewnym ustalonym języku programowania, za pomocą ustalonego kompilatora i uruchomić ją na komputerze z ustalonym procesorem, ustaloną hierarchią pamięci, na ustalonym systemie operacyjnym, na którym pracują równolegle ustalone programy i podać na wejście programowi ustaloną funkcję, której zer szukamy i ustaloną dokładność. Czas, który upłynął od podania danych do otrzymania wyniku można traktować jako szybkość działania metody. To podejście ma pewne wady, takie jak testowanie metody na skończonej liczbie funkcji oraz brak możliwości powtórzenia eksperymentu w dokładnie tych samych warunkach. Jednak jest to pomiar łatwy do wykonania i dający pewne wskazówki co do szybkości działania metody.

Zamiast mierzenia czasu możemy zmierzyć liczbę operacji zmiennoprzecinkowych, które wykonano znajdując zera konkretnej funkcji. To podejście wyklucza wiele czynników, biorących udział w poprzednim pomiarze. Przykładowo, pomijane są realizacje operacji przydziału i zwalniania pamięci, wczytywania danych i wypisywania wyników. Zyskujemy możliwość porównywania metod znajdowania zer przez uruchamianie ich na tych samych danych, lecz na innych komputerach. Jednakże, wciąż jesteśmy ograniczeni do testowania na skończonej liczbie funkcji. 

Lepsze metody zostały wymyślone. Możemy rozpatrzyć zależności między kolejnymi przedziałami dla dowolnej funkcji analizując operację znajdowania następnika przedziału w ciągu lub potomków przedziału w drzewie przedziałów. Posługujemy się pojęciem rzędu zbieżności, czyli miarą tego, ile krótszy jest następny wygenerowany przedział od poprzedniego. Oznaczając długość $k$-tego przedziału w ciągu jako $e_k$ metoda ma rząd zbieżności p, gdy:

$$e_k^p \sim e_{k+1}$$

Biorąc pod uwagę, że przedziały nie mogą mieć ujemnej długości, że zależność może się stabilizować dopiero od pewnego kroku, oraz że analizowana metoda powinna być zbieżna, mamy rząd zbieżności 1 gdy:

$$\lim_{k \rightarrow \inf}\frac{e_{k+1}}{e_k} = u\hspace{1cm}0<u<1$$

oraz rząd zbieżności $p$, $p>1$, gdy:

$$\lim_{k\rightarrow\inf}\frac{e_{k+1}}{e_k^p} = u\hspace{1cm}u>0$$

Przykładowo. bisekcja ma rząd zbieżności 1, gdyż następny przedział jest połową poprzedniego:

\begin{equation}
\begin{cases}
e_{k+1}=\frac{1}{2}e_k \\
\lim_{k \rightarrow \inf}\frac{e_{k+1}}{e_k}=\frac{1}{2}
\end{cases}
\end{equation}

\section{Znajdowanie zer wielomianów}

\subsection{Zera wielomianów niskiego stopnia}

Do szukania zer wielomianów niskich stopni można użyć wzorów analitycznych. 

Wielomian stopnia pierwszego ma jeden pierwiastek, który można otrzymać za pomocą jednego dzielenia i zmiany znaku. Wielomian stopnia drugiego ma zero, jeden lub dwa pierwiastki, które można otrzymać wykonując proste operacje arytmetyczne i obliczając pierwiastek kwadratowy. Wielomian stopnia trzeciego ma jeden, dwa lub trzy pierwiastki, które można otrzymać za pomocą prostych działań arytmetycznych i obliczania pierwiastków drugiego i trzeciego stopnia.

Obliczenie pierwiastka $n$-tego stopnia z liczby $a$ wymaga rozwiązania równania $x^n = a$. Dlatego mimo istnienia analitycznych wzorów na zera wielomianów stopnia 2 i 3, obliczenie ich wymaga zastosowania metod numerycznych. Do obliczenia pierwiastka można użyć wyżej opisanej bisekcji, jednak lepsze rezultaty daje skorzystanie z metody Newtona.

\subsection{Zera wielomianów dowolnego stopnia}

Do znalezienia zer wielomianu stopnia wyższego niż 5 należy użyć metod numerycznych. Tych zer nie da się zapisać za pomocą skończonej liczby prostych operacji arytmetycznych i pierwiastkowania. Znajdowanie zer wielomianów nie jest jednak tak ogólnym zadaniem jak znajdowanie zer funkcji ciągłych. Możemy skorzystać z wielu własności charakterystycznych dla wielomianów. Własności te, takie jak ciągłość, ograniczenie liczby zer rzeczywistych przez stopień wielomianu, łatwość obliczania pochodnej dają narzędzia do konstrukcji specyficznych dla wielomianów metod szukania zer.

\section{Wielomiany w postaci Beziera}

Zbiór wielomianów stopnia nie większego niż $n$ jest przestrzenią wektorową o wymiarze $n$. Każda przestrzeń liniowa ma nieskończoną liczbę baz. Jedną z baz wielomianów jest baza potęgowa ${1,x,x^2,x^3,...,x^n}$. Inną bazą jest baza Bernsteina ${B^n_0(x),...,B^n_n(x)}$, gdzie $B^n_i(x) = (n po k)x^i(1-x)^i$. Wielomian wyrażony w bazie Bernsteina nazywany jest wielomianem w postaci Beziera. Każdy wielomian w bazie potęgowej $P(x) = \sum^n_{i=0}p_i x^i$ można przedstawić w bazie Bernsteina $P(x) = \sum^n_{i=0}b_i B^n_i(x)$. Zależność między wektorem $p = <p_0,...,p_n>$ a wektorem $b = <b_0, ..., b_n$ w formie $p [ m_{ij} ] = b$ nazywana jest macierzą konwersji albo macierzą przejścia.

Postać Beziera ma szereg innych własności niż postać potęgowa. W jednych zastosowaniach sprawdza się postać potęgowa, w innych postać Beziera, w innych sprawdzają się jeszcze inne bazy. Przykładowym zastosowaniem postaci Beziera wielomianu jest grafika komputerowa. Zgodnie z twierdzeniem Weierstrassa (udowodnionym zresztą przy pomocy wielomianów w bazie Bernsteina) krzywa parametryczna opisana parą wielomianów może odwzorowywać dowolny kształt ciągły z dowolną dokładnością. Specjalne krzywe wielomianowe, krzywe, których wielomiany są opisane w postaci Beziera umożliwiają łatwe kształtowanie krzywej.

Niech para wielomianów $X(t) = \sum^n_{i=0}x_i B^n_i(t)$ oraz $Y(t) = \sum^n_{i=0}y_i B^n_i(t)$ opisuje krzywą wielomianową. Punkty $\langle (x_0,y_0), ..., (x_n, y_n) \rangle$ nazywane są punktami kontrolnymi krzywej. Można zaobserwować wizualny związek między kształtem krzywej a położeniem punktów kontrolnych.

!! rysunek wielomianu w postaci Beziera wraz z punktami kontrolnymi
!! rysunek krzywej
!! rysunek powierzchni

\section{Znajdowanie zer wielomianu w postaci Beziera}

Szukanie przecięć krzywej opisanej parą wielomianów z prostą jest zadaniem, który sprowadza się do znajdowania zer wielomianu. W przypadku, gdy krzywa jest opisana wielomianami w formie Beziera to zadanie sprowadza się do szukania zer wielomianu w formie Beziera.

Jedną z możliwości jest konwersja postaci wielomianu z postaci Beziera do potęgowej. Istnieją stabilne numerycznie i efektywne (liniowe w stosunku do stopnia wielomianu) algorytmy konwersji. Istnieją sprawdzone algorytmy algorytmy szukania zer wielomianów w formie potęgowej. Jednakże, specyficzne własności postaci Beziera wielomianu pozwalają na konstrukcję efektywnych i stabilnych numerycznie algorytmów operujących na wielomianach w postaci Beziera bez dodatkowego kroku, jakim jest konwersja baz.

* Dzielenie przedziału. Algorytm de Casteljau. Poddziedzina. Rozszerzona definicja wielomianu Beziera ze względu na dziedzinę.

Metoda bezclip. Aby zdefiniowac bezclip wystarczy zdefiniowac znajdowanie potomkow przedzialu w drzewie przedzialu. PRZecięcie otoczki wypuklej punktow kontrolnych wielomianu z osią OX oraz ew. dzielenie na 2 czesci.

!! Dlaczego na 2 czesci? Bo rzad zbieznosci jest 2. Czy to stwierdzenie jest prawdziwe i czy metode, ktora ma rzad zbieznosci 3 najlepiej podzielic na wiecej czesci?

!! rysunek kolejnych otoczek wypuklych

Szybkosc metody bezclip.

Wykorzystujemy właściwości wielomianu: podziedziena (do nowej otoczki wypuklej)

\section{Aproksymacja wielomianu wielomianem}

Aproksymacja to utworzenie uproszczonego modelu, ktory daje pewne informacje o oryginale i jest latwiejszy do przetworzenia. Tutaj bedziemy aproksymowac wielomian wysokiego stopnia, ktorego zer nie mozna okreslic analitycznie wielomianem stopnia nieskiego, ktorego zera mozemy okreslic analitycznie. Musimy takze okreslic, jak zera wielomianu niskiego stopnia roznia sie od zer oryginalnego wielomianu.

Przyblizenie wielomianu wysokiego stopnia wielomianem niskiego stopnia. ``Redukcja stopnia''. 

Aproksymujac wielomian upraszczamy go, tracac informacje (punkty krańcowe, ) ale jednoczesnie zyskujac mozliwosc latwiejszego przetwarzania go.

Macierz redukcji. Obliczanie macierzy redukcji metodą Lewanowicza-Woźnego.

\section{n-clip}

Dzięki skorzystaniu z dalszych własności wlasnosci postaci Beziera: 
* Maksymalna różnica między wielomianami dzięki łatwemu obliczaniu normy maksymalnej wielomianu różnicy.
* Podnoszenie stopnia.

Dwa wielomiany ograniczające.

Wykorzystuje fakt, ze funkcja, ktorej zer szukamy jest wielomianem. Wykorzystujemy takze pewne wlasnosci wielomianow Beziera tj. podzial wielomianu w punkcje czy poddziedzina. (Potrzebne jest to przy wielokrotnym wykorzystaniu tej samej macierzy redukcji, zauwazmy rozwniez, ze po znalezieniu pierwszego zera nie warto szukac zer wielomianu P(x)/(x-r0), gdyz nie mozna wykorzystac tej samej macierzy redukcji)

\subsection{cubiccliping}

\subsection{podsumowanie}

!! na koncu

\section{Dodatek 1: Operacje na przedziałach}

- odejmowanie przedziałów od siebie



\begin{center}
\begin{figure}
\includegraphics{../tests/between2/graph.pdf}
\end{figure}
\end{center}


\end{document}