\documentclass[11pt,a4paper,oneside]{report}

\usepackage[utf8]{inputenc}
\usepackage[polish]{babel}
\usepackage{polski}
\usepackage{graphics}

\makeatother
\title{Znajdowanie zer wielomianu w postaci Beziera za pomocą przycinania sześciennego}
\author{Maciej Pacut}
\date{Wrocław 2010}


\begin{document}
\maketitle
\newpage

\section{Wstęp}

Jakie zastosowania ma znajdowanie zer? Szukanie przeciec krzywej z prosta. Szukanie najblizszych punktow na prostej.

Niniejsza praca daje pogląd na to, czym jest szukanie zer wielomianu i jak napisać metodę szukania zer.

\section{Szukanie zer}


Szukanie numeryczne - co to znaczy? Zawężanie przedziałów na przykładzie bisekcji. Koniec - gdy przedział jest wystarczająco mały. Rysunek kolejnych przedzialow jeden-pod-drugim. Przykladem na rysunku moze byc obliczanie pierwiastka kwadratowego.

Szukanie wszystkich zer - bisekcja z dzieleniem przedziału na 2, gdy jest zbyt długi.

Inne metody. Szybkosc zbieznosci jest jakoscia metody. 

\section{Szukanie zer wielomianów}

Szukanie zer wielomianu. Wzory Cardano. Wyższego stopnia nalezy suzkac numerycznie. REgula descartes. Twierdzenie Sturma.

http://mathworld.wolfram.com/AbelsImpossibilityTheorem.html

 Mamy informacje - ile jest zer conajwyzej, mamy ciągłość

\section{Wielomiany w formie Beziera}

Definicja. Zastosowanie: krzywe i powierzchnie w programach CAD (takze miejsce wymyslenia). Rysunek krzywej, rysunek powierzchni. W formie potegowej nie ma wyraznego powiazania geometrii wielomianu z jego wspolczynnikami. 

Otoczka wypukła. Dzielenie przedziału. Algorytm de Casteljau. Poddziedzina. Rozszerzona definicja wielomianu Beziera ze względu na dziedzinę.

Metoda bezclip. Szybkosc metody bezclip.

\section{Aproksymacja wielomianu wielomianem}

``Redukcja stopnia''

\section{wykorzystanie aproksymacji}

Aproksymujac wielomian upraszczamy go, tracac informacje ale jednoczesnie zyskujac mozliwosc latwiejszego przetwarzania go.

Maksymalna różnica między wielomianami.

Podnoszenie stopnia.

Dwa wielomiany ograniczające.

\section{Operacje na przedziałach}

- odejmowanie przedziałów od siebie

\begin{center}
\begin{figure}
\includegraphics{../tests/between2/graph.pdf}
\end{figure}
\end{center}


\end{document}