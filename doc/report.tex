\documentclass[11pt,a4paper,oneside]{report}

\usepackage[utf8]{inputenc}
\usepackage[polish]{babel}
\usepackage{polski}
\usepackage{graphics}

\makeatother
\title{Znajdowanie zer wielomianu w postaci Beziera za pomocą przycinania sześciennego}
\author{Maciej Pacut}
\date{Wrocław 2010}


\begin{document}
\maketitle
\newpage

\section{Wstęp}

W jaki sposób zaimplementować szukanie miejsc zerowych

Niniejsza praca daje pogląd na to, czym jest szukanie zer wielomianu i jak napisać metodę szukania zer.

\section{Szukanie zer}


Szukanie numeryczne - co to znaczy? Zawężanie przedziałów na przykładzie bisekcji. Koniec - gdy przedział jest wystarczająco mały.

+ istnieją też inne metody, np. newtona, które tworzą ciąg przybliżeń.

Szukanie wielu zer - bisekcja z dzieleniem przedziału na 2, gdy jest zbyt długi.

\section{Szukanie zer wielomianów}

Szukanie zer wielomianu. Wzory Cardano. Wyższego stopnia nalezy suzkac numerycznie. Mamy informacje - ile jest zer conajwyzej, mamy ciągłość

\section{Wielomiany w formie Beziera}

Definicja. Otoczka wypukła, co daje metodę bezclip.

Dzielenie przedziału. Algorytm de Casteljau. Poddziedzina. Rozszerzona definicja wielomianu Beziera ze względu na dziedzinę.

\section{Aproksymacja wielomianu wielomianem}

``Redukcja stopnia''

\section{wykorzystanie aproksymacji}

Maksymalna różnica między wielomianami.

Podnoszenie stopnia.

Dwa wielomiany ograniczające.

\section{Operacje na przedziałach}

- odejmowanie przedziałów od siebie

\begin{center}
\begin{figure}
\includegraphics{../tests/between2/graph.pdf}
\end{figure}
\end{center}


\end{document}