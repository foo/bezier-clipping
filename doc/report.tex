
\documentclass[11pt,a4paper,oneside]{report}

\usepackage[utf8]{inputenc}
\usepackage[polish]{babel}
\usepackage{polski}
\usepackage{graphics}

\makeatother
\title{Znajdowanie zer wielomianu w postaci Beziera za pomocą przycinania sześciennego}
\author{Maciej Pacut}
\date{Wrocław 2010}


\begin{document}
\maketitle
\newpage

\section{Wstęp}

!!!!!!!!! Wstęp napisz na końcu, powinien on opisywac , co zostanie zawarte w pracy.

Jakie zastosowania ma znajdowanie zer? Szukanie przeciec krzywej z prosta. Szukanie najblizszych punktow na prostej.
Niniejsza praca daje pogląd na to, czym jest szukanie zer wielomianu i jak napisać metodę szukania zer.

\section{Numeryczne znajdowanie zer}

\subsection{Znajdowanie jednego zera}

Metoda znajdowania jednego zera funkcji dostaje argumenty:
\begin{enumerate}
\item pewien opis funkcji f, który umożliwia obliczenie wartości tej funkcji dla dowolnego argumentu
\item przedział początkowy $P_0$, czyli przedział w którym szukamy zera funkcji f; metoda nie znajdzie żadnego zera spoza tego przedziału
\item żądana dokładność $\varepsilon$ znalezienia zera funkcji f
\end{enumerate}

Metoda numerycznego znajdowania jednego zera funkcji f generuje ciąg przedziałów $\langle P_0, P_1, ..., P_n\rangle$ o własnościach:
\begin{enumerate}
\item każdy przedział jest podprzedziałem poprzedniego przedziału:
$$P_k \subset P_{k-1}\hspace{1cm}(1 \leq k \leq n)$$
\item każdy przedział zawiera zero funkcji f z przedziału $P_0$

Adnotacja na dole strony: odnosnie zbioru pustego. Jesli w przedziale P0 nie ma zera funkcji f to stwierdzenie, ze przedzial pusty (porazka) zawiera zero funkcji f z przedzialu P0 jest prawdziwe.

\item ostatni przedział $P_n$ jest pierwszym przedziałem krótszym niż $2 \varepsilon$
\end{enumerate}

Jeśli w przedziale $P_0$ metoda nie znalazła zera, to ostatnim przedziałem w ciągu jest przedział pusty; wtedy wynikiem jest brak zer. W przeciwnym przypadku wynikiem jest środek przedziału $P_n$, który jest odległy od zera funkcji f o conajwyżej $\varepsilon$. W przypadku istnienia zera w przedziale $P_0$, udało się znaleźć zero funkcji f z dokładnością $\varepsilon$.

Przedstawiony został w powyższym schemacie pewien zbiór metod znajdowania jednego zera funkcji f. Metody w tym zbiorze różnią się realizacją funkcji generującej ciąg $\langle P_0, P_1, ..., P_n\rangle$. Generowanie ciągu można opisać w postaci szukania podprzedziału $P_{k+1}$ dla danego przedziału $P_k$.

Przykładem metody znajdowania jednego zera funkcji jest bisekcja. W bisekcji przedział $P_{k+1}$ otrzymuje się dzieląc $P_k$ na pół i jako $P_{k+1}$ wybierając tę połowę, w której jest możliwość wystąpienia zera. W przypadku, gdy w żadnej połowie nie ma możliwości wystąpienia zera, $P_{k+1}$ jest przedziałem pustym i metoda kończy działanie. W przypadku, gdy w obu połowach jest możliwość wystąpienia zera, jako $P_{k+1}$ wybierana jest dowolna z połów. W bisekcji należy określić realizację operacji określającej możliwość wystąpienia zera w danym przedziale (w danej połowie przedziału $P_k$). Przykładową taką operacją jest sprawdzenie, czy funkcja f ma w końcach sprawdzanego przedziału różne znaki.

! rysunek bisekcji

W niniejszej pracy zajmiemy się inną metodą z wyżej przedstawionego zbioru metod szukania zer funkcji. Przykładem metody nie należącej do wyżej opisanego zbioru metod jest metoda Newtona. Znaczącą różnicą między metodą Newtona a metodami opisanymi wyżej jest generowanie ciągu przybliżeń zera funkcji zamiast generowania ciągu przedziałów zawierających zero.

\subsection{Znajdowanie wielu zer}

Zmodyfikujmy powyższy schemat znajdowania jednego zera. Zamiast generowania ciągu przedziałów, metoda szukania wielu zer generuje drzewo przedziałów. Drzewo przedziałów ma własności:
\begin{enumerate}
\item w korzeniu znajduje się przedział początkowy $P_0$
\item każdy potomek jest podprzedziałem swojego rodzica
\item przedział z każdego węzła zawiera zero funkcji f !! to niekoniecznie jest prawda
\item każdy liść $L$ jest pierwszym przedziałem na ścieżce od korzenia do $L$, który ma długość mniejszą niż $2 \varepsilon$
\end{enumerate}

Wynikiem jest zbiór środków niepustych przedziałów znajdujących się w liściach drzewa przedziałów.

!! Rysunek drzewa przedziałów w bisekcji przy dwóch podprzedziałach, rysunek drzewa przedziałów zdegenerowanego do listy
!! Podpis do rysunku: W wypadku 1 zera drzewo moze sie degenerowac do listy, ktora mozna traktowac jako ciąg z poprzedniego schematu.

Przedstawiony został w powyższym schemacie pewien zbiór metod znajdowania jednego zera funkcji f. Metody w tym zbiorze różnią się realizacją funkcji generującej drzewo przedziałów. Generowanie drzewa można opisać w postaci operacji znajdowania zbioru podprzedziałów (potomków) dla danego przedziału (rodzica).

Przykładem metody szukania wielu zer funkcji jest modyfikacja opisanej w poprzednim podrozdziale metody bisekcji. W przypadku gdy w obu połowach dzielonego przedziału jest możliwość wystąpienia zera, obie połowy są określane jako potomkowie przedziału.

!! Zdanie wielokrotnie zlozone

Inną realizacją może być np. w przypadku wielomianów: izolacja zer. Zwykla metoda znajdowania zera ignoruje dodatkowe zera.

\subsection{Szybkość metody znajdowania zer}

!! big mess:

Nalezy odroznic dwia sposoby porownywania szybkosci metody. Pierwsza to zaimplementowanie metody w konkretnym jezyku programowania, za pomoca konkretnego kompilatora i uruchomienie jej na konkretnym procesorze z konkretna iloscia cache'u, na konkretnym systemie operacyjnym, na ktorym pracuja rownolegle konkretne programy i zmierzenie czasu dzialania. Podobna metoda jest zliczenie operacji zmiennoprzecinkowych. Druga to matematyczne okreslenie szybkosci metody. 

W niniejszym rozdziale zajmiemy sie drugim sposobem. Uzycie notacji duze O jest bez sensu, gdyz stosunek liczby operacji do wielkosci danyhc wejsciowych moze byc bardzo rozny. PRzede wszystkim zalezy od funkcji. Przykladowo, funkcje sklejane sa opisywane b. duza liczba danych, jednak szukanie zer moze pominac wiele przedzialow. Dlatego dla pomiaru szybkosci metod znajdowania zer uzywamy rzedu zbieznosci, czyli jak z kazdym krokiem kurczą się przedzialy. Bierzemy pod uwage tylko niektore operacje, tj. obliczenie wartosci funkcji w punkcie czy obliczenie pochodnej, reszte operacji traktujac jako tanie. Krok jest liczony w zaleznosci od liczby operacji znaczacych.

\section{Znajdowanie zer wielomianów}

\subsection{Zera wielomianów stopnia niższego niż 5}

Istnieją wzory analityczne na zera wielomianów niskiego stopnia

Wzory na zera 2 i 3 napisac, 4 wspomniec ze są.

Poprawianie wzorow. Wzory Viete'a. Dodatkowy krok metody newtona.

\subsection{Zera wielomianów dowolnego stopnia}

* Wyższego stopnia nalezy szukac numerycznie
** http://mathworld.wolfram.com/AbelsImpossibilityTheorem.html

Znajdowanie zer wielomianów nie jest jednak tak ogólnym zadaniem jak znajdowanie zer funkcji ciągłych. Możemy skorzystać z weilu własności wielomianów.

Gdy mamy informację, że funkcja, której zer szukamy jest wielomianem, możemy skorzystać z własności wielomianów do:
* przyspieszenia metody
* upewnienia się, że znajdujemy wszystkie zera
** ograniczenie na liczbę zer rzeczywistych - stopień
** ustalenie pewnego przyblizenia poczatkowego
** ciągłość 0..stopien-1 pochodnych



\section{Wielomiany w formie Beziera}

!! wiadomosci z ponizszego paragrafu powiniennes potwierdzic źródłami albo chociaz sprawdzic ich poprawnosc w jakims podreczniku.

Zbiór wielomianów stopnia nie większego niż $n$ jest przestrzenią wektorową o wymiarze $n$. Każda przestrzeń liniowa ma nieskończoną liczbę baz. Jedną z baz wielomianów jest baza potęgowa ${1,x,x^2,x^3,...,x^n}$. Inną bazą jest baza Bernsteina ${B^n_0(x),...,B^n_n(x)}$, gdzie $B^n_i(x) = (n po k)x^i(1-x)^i$. Wielomian wyrażony w bazie Bernsteina nazywany jest wielomianem w formie Beziera. Każdy wielomian w bazie potęgowej $P(x) = \sum^n_{i=0}p_i x^i$ można przedstawić w bazie Bernsteina $P(x) = \sum^n_{i=0}b_i B^n_i(x)$. Zależność między wektorem $p = <p_0,...,p_n>$ a wektorem $b = <b_0, ..., b_n$ w formie $p [ m_{ij} ] = b$ nazywana jest macierzą konwersji albo macierzą przejścia.

Different tasks, different tools.

Zastosowanie: krzywe i powierzchnie w programach CAD (takze miejsce wymyslenia). Rysunek krzywej, rysunek powierzchni. W formie potegowej nie ma wyraznego powiazania geometrii wielomianu z jego wspolczynnikami. 

Postac Beziera ma wiele ciekawych wlasnosci:
* Otoczka wypukła. Krótki dowód jako kombinacja wypukla
* Dzielenie przedziału. Algorytm de Casteljau. Poddziedzina. Rozszerzona definicja wielomianu Beziera ze względu na dziedzinę.

Znjdowanie wielu zer. Dzielenie przedziału na wiecej czesci, gdy moze wystapic wiecej niz 1 zero w przedziale (np. gdy jest zbyt długi, czyli przedzial nie kurczy się tak szybko jak by to wynikało z szybkości zbieżności metody). Tutaj, rząd zbieżności metody to 2, więc dzieliomy na dwie częsi w przypadku, gdy przedzial sie nie skurczyl o polowe.

Metoda bezclip. Szybkosc metody bezclip.
Wykorzystujemy właściwości wielomianu: podziedziena (do nowej otoczki wypuklej)
Metoda pasuje do schematu w/w.

\section{Aproksymacja wielomianu wielomianem}

Bez wglebiania sie w samą aproksymacje? Mozna przedstawic aproksymacje jako zagadnienie geometryczne.

Przyblizenie wielomianu wysokiego stopnia wielomianem niskiego stopnia. ``Redukcja stopnia''. 

Aproksymujac wielomian upraszczamy go, tracac informacje (punkty krańcowe, ) ale jednoczesnie zyskujac mozliwosc latwiejszego przetwarzania go.

Macierz redukcji. Obliczanie macierzy redukcji metodą Lewanowicza-Woźnego.

\section{n-clip}

Maksymalna różnica między wielomianami.

Podnoszenie stopnia.

Dwa wielomiany ograniczające.

Modyfikacja n-clip jest modyfikacja bisekcji. Wykorzystuje fakt, ze funkcja, ktorej zer szukamy jest wielomianem. Wykorzystujemy takze pewne wlasnosci wielomianow Beziera tj. podzial wielomianu w punkcje czy poddziedzina. (Potrzebne jest to przy wielokrotnym wykorzystaniu tej samej macierzy redukcji, zauwazmy rozwniez, ze po znalezieniu pierwszego zera nie warto szukac zer wielomianu P(x)/(x-r0), gdyz nie mozna wykorzystac tej samej macierzy redukcji)

\subsection{cubiccliping}

\subsection{podsumowanie}

!! na koncu

\section{Dodatek 1: Operacje na przedziałach}

- odejmowanie przedziałów od siebie



\begin{center}
\begin{figure}
\includegraphics{../tests/between2/graph.pdf}
\end{figure}
\end{center}


\end{document}