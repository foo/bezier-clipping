\documentclass[11pt,a4paper,oneside]{report}

\usepackage[utf8]{inputenc}
\usepackage[polish]{babel}
\usepackage{polski}
\usepackage{graphics}

\makeatother
\title{Znajdowanie zer wielomianu w postaci Beziera za pomocą przycinania sześciennego}
\author{Maciej Pacut}
\date{Wrocław 2010}


\begin{document}
\maketitle
\newpage

\section{Wstęp}

!!!!!!!!! Wstęp napisz na końcu, powinien on opisywac , co zostanie zawarte w pracy.

Jakie zastosowania ma znajdowanie zer? Szukanie przeciec krzywej z prosta. Szukanie najblizszych punktow na prostej.
Niniejsza praca daje pogląd na to, czym jest szukanie zer wielomianu i jak napisać metodę szukania zer.

\section{Szukanie zer}

Zajmijmy się metodą szukania zer, która przebiega wg następującego schematu:
* Znalezienie zera numerycznie to znalezienie przedzialu o dlugosci <= \eps, ktory zawiera zero (zastanow sie, czy na pewno - w cubicclip zero moze nie wystapic, jesli pierwiastek jest parzystokrotny).
* Metoda szukania zer generuje ciąg przedziałów, z których kazdy nastepny jest podprzedizalem poprzedniego.
* koniec - gdy przedział jest wystarczająco mały

Dobrym przykładem metody tego typu jest bisekcja. Przykladem na rysunku moze byc obliczanie pierwiastka kwadratowego.

Szukanie wszystkich zer rzeczywistych - bisekcja z dzieleniem przedziału na 2 gdy moze wystapic wiecej niz 1 zero w przedziale (np. gdy jest zbyt długi). Modyfikacja schematu.

Inne metody. Szybkosc zbieznosci jest jakoscia metody. 

\section{Szukanie zer wielomianów}

Wzory analityczne na zera wielomianów niskiego stopnia (2 i 3 napisac, 4 wspomniec ze są)
* Wyższego stopnia nalezy szukac numerycznie
** regula descartes
** twierdzenie Sturma.
** http://mathworld.wolfram.com/AbelsImpossibilityTheorem.html


Gdy mamy informację, że funkcja, której zer szukamy jest wielomianem, możemy skorzystać z własności wielomianów do:
* przyspieszenia metody
* upewnienia się, że znajdujemy wszystkie zera
** ograniczenie na liczbę zer rzeczywistych - stopień
** ustalenie pewnego przyblizenia poczatkowego



\section{Wielomiany w formie Beziera}

Baza Bernsteina. Również jest bazą wielomianów, co oznacza, że każdy wielomian w bazie potęgowej ma odpowiadające przedstawienie w bazei Bernsteina.

Zastosowanie: krzywe i powierzchnie w programach CAD (takze miejsce wymyslenia). Rysunek krzywej, rysunek powierzchni. W formie potegowej nie ma wyraznego powiazania geometrii wielomianu z jego wspolczynnikami. 

Postac Beziera ma wiele ciekawych wlasnosci:
* Otoczka wypukła. Krótki dowód jako kombinacja wypukla
* Dzielenie przedziału. Algorytm de Casteljau. Poddziedzina. Rozszerzona definicja wielomianu Beziera ze względu na dziedzinę.

Metoda bezclip. Szybkosc metody bezclip.
Wykorzystujemy właściwości wielomianu: podziedziena (do nowej otoczki wypuklej)
Metoda pasuje do schematu w/w.

\section{Aproksymacja wielomianu wielomianem}

Bez wglebiania sie w samą aproksymacje? Mozna przedstawic aproksymacje jako zagadnienie geometryczne.

Przyblizenie wielomianu wysokiego stopnia wielomianem niskiego stopnia. ``Redukcja stopnia''. 

Aproksymujac wielomian upraszczamy go, tracac informacje (punkty poczatkowe, ) ale jednoczesnie zyskujac mozliwosc latwiejszego przetwarzania go.

\section{n-clip}

Maksymalna różnica między wielomianami.

Podnoszenie stopnia.

Dwa wielomiany ograniczające.

Modyfikacja n-clip jest modyfikacja bisekcji. Wykorzystuje fakt, ze funkcja, ktorej zer szukamy jest wielomianem. Wykorzystujemy takze pewne wlasnosci wielomianow Beziera tj. podzial wielomianu w punkcje czy poddziedzina. (Potrzebne jest to przy wielokrotnym wykorzystaniu tej samej macierzy redukcji)

\subsection{cubiccliping}

\subsection{podsumowanie}

!! na koncu

\section{Dodatek 1: Operacje na przedziałach}

- odejmowanie przedziałów od siebie



\begin{center}
\begin{figure}
\includegraphics{../tests/between2/graph.pdf}
\end{figure}
\end{center}


\end{document}