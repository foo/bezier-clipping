
\documentclass[11pt,a4paper,oneside]{report}

\usepackage[utf8]{inputenc}
\usepackage[polish]{babel}
\usepackage{polski}
\usepackage{graphics}

\makeatother
\title{Znajdowanie zer wielomianu w postaci Beziera za pomocą przycinania sześciennego}
\author{Maciej Pacut}
\date{Wrocław 2010}


\begin{document}
\maketitle
\newpage

\section{Wstęp}

!!!!!!!!! Wstęp napisz na końcu, powinien on opisywac , co zostanie zawarte w pracy.

Jakie zastosowania ma znajdowanie zer? Szukanie przeciec krzywej z prosta. Szukanie najblizszych punktow na prostej.
Niniejsza praca daje pogląd na to, czym jest szukanie zer wielomianu i jak napisać metodę szukania zer.

\section{Numeryczne znajdowanie zer}

\subsection{Znajdowanie jednego zera}

Metoda znajdowania jednego zera funkcji dostaje argumenty:
\begin{enumerate}
\item pewien opis funkcji f, który umożliwia obliczenie wartości tej funkcji dla dowolnego argumentu
\item przedział początkowy $P_0$, czyli przedział w którym szukamy zera funkcji f; metoda nie znajdzie żadnego zera spoza tego przedziału
\item żądana dokładność $\varepsilon$ znalezienia zera funkcji f
\end{enumerate}

Zakładając, że w przedziale $P_0$ znajduje się zero funkcji f, metoda numerycznego znajdowania jednego zera funkcji f generuje ciąg przedziałów $\langle P_0, P_1, ..., P_n\rangle$ o własnościach:
\begin{enumerate}
\item każdy przedział jest podprzedziałem poprzedniego przedziału:
$$P_k \subset P_{k-1}\hspace{1cm}(1 \leq k \leq n)$$
\item każdy przedział zawiera zero funkcji f
\item ostatni przedział $P_n$ jest pierwszym przedziałem krótszym niż $2 \varepsilon$
\end{enumerate}

Wynikiem jest środek przedziału $P_n$, który jest odległy od zera funkcji f o conajwyżej $\varepsilon$. W przypadku istnienia zera w przedziale $P_0$, udało się znaleźć zero funkcji f z dokładnością $\varepsilon$.

Przedstawiona została w powyższym schemacie pewien zbiór metod znajdowania jednego zera funkcji f. Metody w tym zbiorze różnią się realizacją funkcji generującej ciąg $\langle P_0, P_1, ..., P_n\rangle$.
Przykładem metody znajdowania jednego zera funkcji jest bisekcja. W bisekcji przedział $P_{k+1}$ otrzymuje się dzieląc $P_k$ na pół i jako $P_{k+1}$ wybierając tę połowę, w której jest możliwość wystąpienia zera. W bisekcji należy określić realizację operacji określającej możliwość wystąpienia zera w danym przedziale (w danej połowie przedziału $P_k$). Przykładową taką operacją jest sprawdzenie, czy funkcja f ma w końcach sprawdzanego przedziału różne znaki.

! rysunek bisekcji

W niniejszej pracy zajmiemy się inną metodą z wyżej przedstawionego zbioru metod szukania zer funkcji. Przykładem metody nie należącej do wyżej opisanego zbioru metod jest metoda Newtona. Znaczącą różnicą między metodą Newtona a metodami opisanymi wyżej jest generowanie ciągu przybliżeń zera funkcji zamiast generowania ciągu przedziałów zawierających zero.

\subsection{Znajdowanie wielu zer}

Zmodyfikujmy powyższy schemat znajdowania jednego zera. Zamiast generowania ciągu przedziałów, metoda szukania wielu zer generuje drzewo przedziałów. Drzewo przybliżeń ma własności:
\begin{enumerate}
\item w korzeniu znajduje się przedział początkowy $P_0$
\item każdy potomek jest podprzedziałem swojego rodzica
\item zawieranie zer na kazdej sciezce
\item liście drzewa są pierwszymi przedziałami na sciezce do korzenia, które mają długosc mniejsza niz 2 eps
\end{enumerate}

Niektore sciezki moga sie konczyc przedzialami dluzszymi niz 2 eps, wtedy sa to porazki.

!! zmodyfikuj schemat 1 zera tak, aby mial mozliwosc porazki (ost przedzial dluzszy niz 2 eps).

W wypadku 1 zera drzewo moze sie degenerowac do listy, ktora mozna traktowac jako ciąg z poprzedniego schematu.

Przykład: niescislosc w definicji bisekcji.

Dzielenie przedziału na wiecej czesci, gdy moze wystapic wiecej niz 1 zero w przedziale (np. gdy jest zbyt długi, czyli przedzial nie kurczy się tak szybko jak by to wynikało z szybkości zbieżności metody).

Inną realizacją może być np. w przypadku wielomianów: izolacja zer. Zwykla metoda znajdowania zera ignoruje dodatkowe zera.

\subsection{Jakość metody znajdowania zer}

Szybkosc zbieznosci jest jakoscia metody. Jak z kazdym krokiem kurczą się przedzialy. Krok jest liczony w zaleznosci od liczby operacji znaczacych tj. obliczenie wartosci funkcji.

\section{Znajdowanie zer wielomianów}

Wzory analityczne na zera wielomianów niskiego stopnia (2 i 3 napisac, 4 wspomniec ze są)
* Wyższego stopnia nalezy szukac numerycznie
** http://mathworld.wolfram.com/AbelsImpossibilityTheorem.html


Gdy mamy informację, że funkcja, której zer szukamy jest wielomianem, możemy skorzystać z własności wielomianów do:
* przyspieszenia metody
* upewnienia się, że znajdujemy wszystkie zera
** ograniczenie na liczbę zer rzeczywistych - stopień
** ustalenie pewnego przyblizenia poczatkowego



\section{Wielomiany w formie Beziera}

Baza Bernsteina. Również jest bazą wielomianów, co oznacza, że każdy wielomian w bazie potęgowej ma odpowiadające przedstawienie w bazei Bernsteina.

Zastosowanie: krzywe i powierzchnie w programach CAD (takze miejsce wymyslenia). Rysunek krzywej, rysunek powierzchni. W formie potegowej nie ma wyraznego powiazania geometrii wielomianu z jego wspolczynnikami. 

Postac Beziera ma wiele ciekawych wlasnosci:
* Otoczka wypukła. Krótki dowód jako kombinacja wypukla
* Dzielenie przedziału. Algorytm de Casteljau. Poddziedzina. Rozszerzona definicja wielomianu Beziera ze względu na dziedzinę.

Metoda bezclip. Szybkosc metody bezclip.
Wykorzystujemy właściwości wielomianu: podziedziena (do nowej otoczki wypuklej)
Metoda pasuje do schematu w/w.

\section{Aproksymacja wielomianu wielomianem}

Bez wglebiania sie w samą aproksymacje? Mozna przedstawic aproksymacje jako zagadnienie geometryczne.

Przyblizenie wielomianu wysokiego stopnia wielomianem niskiego stopnia. ``Redukcja stopnia''. 

Aproksymujac wielomian upraszczamy go, tracac informacje (punkty krańcowe, ) ale jednoczesnie zyskujac mozliwosc latwiejszego przetwarzania go.

Macierz redukcji. Obliczanie macierzy redukcji metodą Lewanowicza-Woźnego.

\section{n-clip}

Maksymalna różnica między wielomianami.

Podnoszenie stopnia.

Dwa wielomiany ograniczające.

Modyfikacja n-clip jest modyfikacja bisekcji. Wykorzystuje fakt, ze funkcja, ktorej zer szukamy jest wielomianem. Wykorzystujemy takze pewne wlasnosci wielomianow Beziera tj. podzial wielomianu w punkcje czy poddziedzina. (Potrzebne jest to przy wielokrotnym wykorzystaniu tej samej macierzy redukcji, zauwazmy rozwniez, ze po znalezieniu pierwszego zera nie warto szukac zer wielomianu P(x)/(x-r0), gdyz nie mozna wykorzystac tej samej macierzy redukcji)

\subsection{cubiccliping}

\subsection{podsumowanie}

!! na koncu

\section{Dodatek 1: Operacje na przedziałach}

- odejmowanie przedziałów od siebie



\begin{center}
\begin{figure}
\includegraphics{../tests/between2/graph.pdf}
\end{figure}
\end{center}


\end{document}